\documentclass[oneside]{amsart}
\usepackage[latin9]{inputenc}
\usepackage{geometry,amsmath,xcolor,fancybox,multicol,afterpage,termcal}

\usepackage{lipsum}

\newcommand\textbox[1]{%
  \parbox{.333\textwidth}{#1}%
}

\makeatletter
   \def\vhrulefill#1{\leavevmode\leaders\hrule\@height#1\hfill \kern\z@}
\makeatother




\newtheorem{theorem}{Theorem}[section]
\newtheorem{corollary}{Corollary}[theorem]
\newtheorem{proposition}[theorem]{Proposition}
\newtheorem*{lemma}{Lemma}
\theoremstyle{definition}
\newtheorem*{definition}{Definition}
\theoremstyle{definition}
\newtheorem{example}{Example}

\usepackage{pgfplots}
\pgfplotsset{%
   every tick label/.append style = {font=\tiny},
   every axis label/.append style = {font=\scriptsize}
}

\usepackage{tikz}
\tikzset{
  jumpdot/.style={mark=*,solid},
  excl/.append style={jumpdot,fill=white},
  incl/.append style={jumpdot,fill=black},
}

\geometry{verbose,tmargin=1.5cm,bmargin=1.5cm,lmargin=1.5cm,rmargin=1.5cm}

\usepackage{zref-savepos}
\makeatletter
% \zsaveposx is defined since 2011/12/05 v2.23 of zref-savepos
\@ifundefined{zsaveposx}{\let\zsaveposx\zsavepos}{}
\makeatother
\newcounter{hposcnt}
\renewcommand*{\thehposcnt}{hpos\number\value{hposcnt}}
\newcommand*{\SP}{% set position
  \stepcounter{hposcnt}%
  \zsaveposx{\thehposcnt s}%
}
\makeatletter
\newcommand*{\UP}{% use previous position
  \zsaveposx{\thehposcnt u}%
  \zref@refused{\thehposcnt s}%
  \zref@refused{\thehposcnt u}%
  \kern\zposx{\thehposcnt s}sp\relax
  \kern-\zposx{\thehposcnt u}sp\relax
}
\makeatother


\thispagestyle{empty}

\begin{document}
Probability \hfill {\huge WS 7} \hfill Due: September $22^{nd}$ start of class \\
\begin{flushright}
    {\Large Name:}\rule[-1mm]{75mm}{.1mm}
\end{flushright}

 \hrulefill
\vspace{2mm}

\begin{ovalbox}{\begin{minipage}{6.8in}
\vspace{5mm}

\textbf{Definition:} The probability of event $A$ occurring given that event $B$ occurred is called \textbf{conditional probability}, written $P(A|B)$ and is computed via \\
\begin{center}
  $P(A|B) =$ \rule[-5mm]{35mm}{.1mm}
\end{center}

\end{minipage}}
\end{ovalbox}

\vspace{3mm}


\textbf{Example:} What is the probability that a card drawn from a deck is a jack given that it is a heart?

\vfill

\textbf{Example:} Your neighbors have two kids and you know that one of them is a girl. What is the probability that their second child is a girl?

\vfill



The definition of two events being independent is that $P(A \text{ and } B) = P(A)P(B)$. Make this substitution into the conditional probability formula to get a new characterization of independent events:

\begin{ovalbox}{\begin{minipage}{6.8in}
\vspace{5mm}

Suppose events $A$ and $B$ are independent, then: \\
\begin{center}
  $P(A|B) =$ \rule[-5mm]{35mm}{.1mm}
\end{center}

\end{minipage}}
\end{ovalbox}

\vspace{5mm}

\newpage

\textbf{Example:} You are buying a used car in city where rainfall causes street flooding often. You know that $5 \%$ of used cars have been damaged from flooding and $80 \%$ of those cars will later experience serious engine problems. On the other hand, only $10 \%$ of cars without flood damage will experience the same engine issues. What is the probability the car you buy will later experience engine issues?

\vfill
\vfill

\vhrulefill{2pt}

\vspace{1mm}
\vhrulefill{2pt}
\\

{\Large \textbf{ Class Activity}:}
\\
\begin{enumerate}
 \item[1.] Suppose you flip a fair coin twice:
    \begin{enumerate}
        \item What is the probability of \textbf{Event A:} getting at most one heads?
        \vfill
        \item What is the probability of \textbf{Event B:} getting two of the same result (both heads or both tails)?
        \vfill
        \item What is the probability of \textbf{Event C:} getting heads on the first flip and tails on the second?
        \vfill
        \item Are events A and B disjoint? Are they independent? 
        \vfill
        \item Are events A and C disjoint? Are they independent? 
        \vfill
        \item Are events B and C disjoint? Are they independent? 
        \vfill
\end{enumerate}

    \newpage

    \item[2.] Suppose you roll two dice: 
    \begin{enumerate}
        \item What is the probability that you roll doubles?

        \vfill

        \item In the board game Monopoly, if you roll doubles you get to roll again. However, if you roll doubles three times in a row you are sent to jail. If you've rolled doubles twice, what is the probability that you will get doubles on your next roll?
   
        \vfill

        \item Is the event of rolling the dice the third time disjoint from the event of rolling the dice the second time? Are they independent? 

        \vfill

        \item What is the probability of rolling doubles three times in a row? 

        \vfill

    \end{enumerate}
   

    \item[3.] You have a box with two balls in it, one red and one blue. We select one ball from the box, put it back and select another. 
    \begin{enumerate}
        \item Let's say event $R$ is the event where you get the red ball twice, what is $P(R)$?

        \vfill

        \item Let's say event $F$ is the event that you get the red ball on your first pull, what is $P(R|F)$?

        \vfill

        \item Are $R$ and $F$ independent? Are they disjoint?

        \vfill

        \item Let $S$ be the event that you pull the red ball on the second pick. Are $F$ and $S$ independent? Are they disjoint?

    \end{enumerate}
    
    \vfill

    \newpage
    
        \item[4.] Consider the following game where there are three dice with sides:

        \begin{align*}
            \text{Die }A&: \{1,1, 5, 5, 5, 5 \}  \\
            \text{Die }B&: \{3,3,3, 4, 4, 4 \}  \\
            \text{Die }C&: \{2,2,2, 2, 6, 6 \}  \\
        \end{align*}
        The game is as follows: two players take turns selecting a die and whoever rolls the highest number wins.
        \begin{enumerate}
            \item What is the probability that Die $A$ beats Die $B$?

            \vfill 
            \item What is the probability that Die $B$ beats Die $C$?

            \vfill 

            \item What is the probability that Die $C$ beats Die $A$?

            \vfill 

            \item What can you do to maximize your odds of winning the game?

            \vfill 
        \end{enumerate}
        

\newpage

                 \item[5.] A swim team has 150 members. On the team there are 75 advanced swimmers, 47 intermediate swimmers, and the rest are novice swimmers. Many swimmers practice 5 days a week: 40 of the advanced, 30 intermediate, and 10 novices.

             \begin{enumerate}
                 \item What is the probability that a randomly chosen swimmer is a novice?
                 \vfill
                 \item What is the probability that a randomly chosen swimmer is advanced and practices 5 times a week?
                 \vfill
                 \item Is being advanced and practicing 4 times a week disjoint events? Are they independent?
                 \vfill
             \end{enumerate}

                              \item[6.] On a given day the probability that I go to a coffee shop is $P(CS)=.2$, the probability that I play chess is $P(Chess)=.5$, and $P(Chess|CS)=.8$.

             \begin{enumerate}
                 \item Find $P(CS$ and $Chess)$.
                 \vfill
                 \item Find $P(CS$ or $Chess)$.
                 \vfill
             \end{enumerate}
        
    

    
\end{enumerate}






\end{document}