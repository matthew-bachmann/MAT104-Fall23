\documentclass[oneside]{amsart}
\usepackage[latin9]{inputenc}
\usepackage{geometry,amsmath,xcolor,fancybox,multicol,afterpage,termcal}

\usepackage{lipsum}

\newcommand\textbox[1]{%
  \parbox{.333\textwidth}{#1}%
}

\makeatletter
   \def\vhrulefill#1{\leavevmode\leaders\hrule\@height#1\hfill \kern\z@}
\makeatother




\newtheorem{theorem}{Theorem}[section]
\newtheorem{corollary}{Corollary}[theorem]
\newtheorem{proposition}[theorem]{Proposition}
\newtheorem*{lemma}{Lemma}
\theoremstyle{definition}
\newtheorem*{definition}{Definition}
\theoremstyle{definition}
\newtheorem{example}{Example}

\usepackage{pgfplots}
\pgfplotsset{%
   every tick label/.append style = {font=\tiny},
   every axis label/.append style = {font=\scriptsize}
}

\usepackage{tikz}
\tikzset{
  jumpdot/.style={mark=*,solid},
  excl/.append style={jumpdot,fill=white},
  incl/.append style={jumpdot,fill=black},
}

\geometry{verbose,tmargin=1.5cm,bmargin=1.5cm,lmargin=1.5cm,rmargin=1.5cm}

\usepackage{zref-savepos}
\makeatletter
% \zsaveposx is defined since 2011/12/05 v2.23 of zref-savepos
\@ifundefined{zsaveposx}{\let\zsaveposx\zsavepos}{}
\makeatother
\newcounter{hposcnt}
\renewcommand*{\thehposcnt}{hpos\number\value{hposcnt}}
\newcommand*{\SP}{% set position
  \stepcounter{hposcnt}%
  \zsaveposx{\thehposcnt s}%
}
\makeatletter
\newcommand*{\UP}{% use previous position
  \zsaveposx{\thehposcnt u}%
  \zref@refused{\thehposcnt s}%
  \zref@refused{\thehposcnt u}%
  \kern\zposx{\thehposcnt s}sp\relax
  \kern-\zposx{\thehposcnt u}sp\relax
}
\makeatother


\thispagestyle{empty}

\begin{document}
Probability \hfill {\huge WS 6} \hfill Due: September $18^{th}$ start of class \\
\begin{flushright}
    {\Large Name:}\rule[-1mm]{75mm}{.1mm}
\end{flushright}

 \hrulefill
\vspace{2mm}

    \textbf{Example:} Consider the experiment of choosing 2 marbles from a bag with 2 red marbles, 2 blue marble, and 1 green marble. We will describe the probability of all events from the sample space $S = \{ RR, RB, RG, BB, BG \}$

    \vfill

\begin{ovalbox}{\begin{minipage}{6.8in}
\vspace{5mm}

\textbf{Definition:} A table of all probabilities for a sample space is called a  \\
\begin{center}
   \rule[-1mm]{85mm}{.1mm}
\end{center}


\vspace{7mm}

\end{minipage}}
\end{ovalbox}

\vspace{3mm}

\textbf{Properties:}
\begin{itemize}
    \item Each outcome must be disjoint
    \item The probabilities must be between $0$ and $1$
    \item The probabilities add to $1$.
\end{itemize}

\vspace{3mm}

\begin{ovalbox}{\begin{minipage}{6.8in}
\vspace{5mm}

\textbf{Definition:} When all of the outcomes are equally likely we say that the probability distribution is   \\
\begin{center}
   \rule[-1mm]{85mm}{.1mm}
\end{center}


\vspace{7mm}

\end{minipage}}
\end{ovalbox}

\vspace{2mm}

Recall that when all outcomes are equally likely the probability of an event occurring is


$$ P(E) = \displaystyle \frac{\# \text{ favorable outcomes}}{\# \text{ possible outcomes}}$$

\vspace{2mm}


\begin{ovalbox}{\begin{minipage}{6.8in}
\vspace{5mm}

\textbf{Definition:} The \textbf{complement} of a subset are all elements in the set that are not in the subset.   \\

\vspace{7mm}

\end{minipage}}
\end{ovalbox}

\vspace{5mm}

   \textbf{Example}: Let $S = \{a,b,c,d,e,f,g \}$ and $A = \{a,b \}$ then what is $A^c$ (the complement of $A$)?
    
    \vfill


\begin{ovalbox}{\begin{minipage}{6.8in}
\vspace{5mm}

The complement of a set has the property that  \\
\begin{center}
  $P(A \text{ or } A^c)= $\rule[-2mm]{35mm}{.1mm} $=$\rule[-2mm]{35mm}{.1mm}.
\end{center}
\vspace{2mm}
That is, a subset is always disjoint from its complement and the probability that one or the other happens is $1$.
\vspace{7mm}

\end{minipage}}
\end{ovalbox}

\vspace{3mm}

\newpage

Sometimes it is easier to compute the probability of the complement of an event and then use the above property to compute the probability of the event.

    \textbf{Example:} What is the probability of two people in our class having the same birthday? (there are 20 people in our class)
    
    \vfill

    \textbf{Example:} There is a $50 \%$ chance of rain on Saturday and a $50 \%$ chance of rain on Sunday. Someone (not in our class) concludes that it will definitely rain this weekend since

    $$P(\text{rain on Saturday or rain on Sunday}) = P(\text{rain on Saturday}) + P(\text{rain on Sunday}) = 100 \%.$$ 
    Why is their calculation wrong? How can we correctly compute the probability?
    \vfill


\begin{ovalbox}{\begin{minipage}{6.8in}
\vspace{5mm}

\textbf{Definition:} Two events are said to be \textbf{independent} if the outcome of one does not affect the probability of the other. The probability of independent events $A$ and $B$ occurring is  \\
\begin{center}
  $P(A \text{ and } B)= $\rule[-2mm]{35mm}{.1mm}.
\end{center}

\vspace{7mm}

\end{minipage}}
\end{ovalbox}

\vspace{2mm}
Independent events can sometimes be difficult to recognize, consider the following:

\textbf{Example:} What is the probability of drawing a card that it is both $A=\{$ a heart $\}$ and $B= \{$ an ace $\}$? Are the events independent? What if we remove the 2 of spades from the deck?

\vfill


\newpage

\vhrulefill{2pt}

\vspace{1mm}
\vhrulefill{2pt}
\\

{\Large \textbf{ Class Activity}:}
\\

\begin{itemize}
    \item[1.] Describe the probability distributions for the following random experiments: 
    \begin{itemize}
        \item  Flipping a fair coin twice.

        \vfill

        \item Counting the number of heads from flipping a coin three times.
   
        \vfill

        \item The sum of rolling two six sided dice.

        \vfill

    \end{itemize}
   

    \item[2.] What is the probability of rolling doubles from three dice?
    
    \vfill

    
        \item[3.] Assume that there is a one in a million chance that someone is struck by lightning if they are outside during a storm. If there are $50,000$ people outside in a city during a storm, what is the chance that someone is struck by lightning? 
    
    \vfill

\newpage
        \item[4.] Let $A$ and $B$ be events with $P(A)=\frac{1}{2}$ and $P(B)=\frac{1}{3}$. Find $P( A \text{ or } B)$ and $P(A \text{ and }B)$ if:
        \begin{itemize}
            \item $A$ and $B$ are disjoint.
            \vfill

            \item $A$ and $B$ are independent.

            \vfill

            \item $A^c$ and $B$ are independent

            \vfill
        \end{itemize}

    \item[5.] Roll a die twice and consider the events $A = \{ \text{first roll gives at least 4} \}$, $B = \{ \text{second roll gives at most 4} \}$, and $C = \{ \text{the sum of the rolls is 10\}}$. Find $P(A)$, $P(B)$, $P(C)$, and $P(A \cap B \cap C)$. Are $A$ and $B$ independent? What about $B$ and $C$? Finally, are $A$ and $C$ independent? (we call this \textbf{pairwise independence})
    
    \vfill
    

    
\end{itemize}






\end{document}